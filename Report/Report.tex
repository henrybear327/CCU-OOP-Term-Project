%!TEX TS-program = xelatex
%!TEX encoding = UTF-8 Unicode

% sudo easy_install Pygments
% Add -shell-escape to compile flag

% Options -> Editor -> Indentation mode -> replace tab by space!! -> avoids ^^I issue in code

\documentclass[12pt, A4]{article}
\usepackage[utf8]{inputenc}
\usepackage[linguistics]{forest}
\usepackage{listings}

%%%%%% Font setting %%%%%%

\usepackage{fontspec} %加這個就可以設定字體 

\usepackage{xeCJK} %讓中英文字體分開設置

%\setCJKmainfont{Noto Sans CJK TC} %設定中文為系統上的字型,而英文不去更動,使用原TeX字型
\setCJKmainfont{STKaiti}

\XeTeXlinebreaklocale "zh" %這兩行一定要加,中文才能自動換行
\XeTeXlinebreakskip = 0pt plus 1pt 

%%%%%% Coding style %%%%%%
\usepackage{minted}
\usemintedstyle{friendly}
\setminted[c]{
    linenos=true,
    autogobble=true,
}

\usepackage{hyperref}

%%%%%% 文件正式開始 %%%%%%

\title{Object-Oriented Programming\\ Assignment 5\\ Project}
\author{403410033 \ 資工三 \ 曾俊宏}
\date{\today}

\begin{document}
	
	\maketitle
	%\newpage

    \section{Concurrent Development}
    
    Concurrent development sounds good in theory, but in reality it's pretty hard to achieve.\\
    
    By defining all of the interfaces first, then starts writing code, we can divide the work among teammates. We can call the method even if it's not implemented yet, knowing that it'll eventually be working in the end.\\
    
    Having interfaces also have an advantage that we can easily swap the implementations. For example, in our team, class Rational have 2 different implementations. We can easily swap them and test for correctness.\\
    
    The downside of concurrent development is that teammates are procrastinating too much. When this happens, the development will be slowed down significantly, since testing can't be done at all. \\
    
    Take our team for example, no one was working on the project when I am already on it. So I decided to implement all interfaces and write a testbench. Testbench is a pretty good idea for making you confident that all of the code your teammate eventually turnin won't break the codebase, especially when they are under deadline pressure.
    
    \section{What are the advatanges of OOP principles (encapsulation, inheritance, and polymorphism) used in this work?}
    
    Encapsulation: The implementation details are hidden from the user. When Rational class uses BigInt, it doesn't need to know, for example, how addition is implemented. All we need to know is how to correctly use it, e.g. how to invoke BigInt methods, what will be returned. The major advantage is that everybody just need to take care of their respective jobs, and trust the components that you use will give you the correct result.\\
    
    Inheritance: Not used.\\
    
    Polymorphism: Not used.
    
    \section{What did you learn from this experience of team programming?}
    
    Testbench is pretty awesome! It gives you confidence as you develop and make modification to the code you wrote. Better yet, when your teammate turn in their code, you can test it quickly and if all the tests passed you can be very certain that the code they wrote is correct.
    
    \section{Teammate Evaluation}
    
    \subsection{黃子俞}
    
    Score: 10\\
    
    An average teammate. \\
    
    Accomplished tasks, but it's too close to deadline so there isn't a lot of time to discuss with him. \\
    
    \subsection{童彥淇}
    
    Score: 5\\
    
    An average teammate. \\
    
    She's in charge of the BigInt part, which is the core of the entire program. But she started working on the day before deadline, which is a pretty bad idea.\\ 
    
    Worse yet, the code she turned in was incorrect. (To be exact, only the addition is correct. I am kind of glad that I decided to take on her part myself in the beginning, or our team won't have functional codebase at all)
    
    \section{Notes on code}
    
    The codebase is fully tested on MacOS 10.12.5, using g++ compiler 7.1.0\\
    
    \subsection{Makefile}
    
    $make\ normal$: Produces an executable file named $normal$, which runs the sample test case provided on the spec.\\
    
    $make\ test$: Produces an executable file named $test$. This is the testbench that I used for testing the correctness of codebase.
    
    \subsection{Codebase}
    
    The folder $Teammate$ contains the code written by my teammates. The code in $Rational$ folder is working and correct (authored by 黃子俞). The code in $BigInt$ folder is incorrect, although it can be compiled successfully (authored by 童彥淇).\\
    
    The folder $Codebase$ contains the final version of our project. Compile using the aforementioned method should work. \\
    
    The code file $main.cpp$ contains the sample test case.
	
\end{document}